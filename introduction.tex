\chapter{บทนำ}

\par{
ก่อนจะถึงในรายละเอียดข้างต้น 
ในส่วนแรกจะกล่าวถึงประวัติและความเป็นมาของ
ระบบการคำนวณอันเป็นที่มาของสถาปัตยกรรมคอมพิวเตอร์
}

\par{
Charles Babbage เป็นศาสตราจารย์ในสาขาคณิตศาสตร์ที่มหาวิทยาลัย 
Cambridge ในช่วง 1827 ถึง 1839 ซึ่งถูกยกย่องว่า
เป็นบิดาแห่งคอมพิวเตอร์ (father of the computer) 
\cite{halacy1970charles}
เนื่องจากเป็นผู้คิดค้นเครื่องคำนวณแบบจักรกลเป็นผลสำเร็จเป็นคนแรก อัน
นำมาซึ่งในการออกแบบเครื่องคำนวณในรูปแบบต่าง ๆ ที่มีความซับซ้อนมากยิ่งขึ้น 
หนึ่งในเครื่องจักรการคำนวณที่จะกล่าวถึงคือ Different Engine 
(ถูกออกแบบในปี 1823) ดังแสดงในรูปที่ 
\ref{fig_babbage_difference_engine}
}
%
%
\begin{figure}[h]
\centering
\includegraphics[width=0.75\textwidth]{fig/Babbage_Difference_Engine.png}
\caption{Different Engine ที่ถูกจัดแสดงใน London Science Museum}
\label{fig_babbage_difference_engine}
\end{figure}
%
%
\par{
จุดมุ่งหมายของการออกแบบ Different Engine 
คือต้องการใช้เครื่องจักรในการแก้ปัญหาทางคณิตศาสตร์ที่มีหลัก
ในการแก้เป็นแบบการวนซ้ำ (Iteration) ซึ่งปัญหาคณิตศาสตร์
ดังกล่าวคือ การหาค่าของสมการพหุนามด้วย 
Finite Difference Method ตัวอย่างเช่น  
กำหนดให้  $f(x) เป็นฟังก์ชั่นพหุนาม โดยที่มีค่าเป็น 
$x^2 + x + 33$ ดังนั้น
%
\begin{align}
f(x)&=x^2+x+33\\
d_1(x)&=f(x) - f(x-1) = 2x\\
d_2(x)&=d_1(x) - d_1(x-1) = 2
\end{align}
%
ในทางย้อนกลับจากข้างต้นจะได้ว่า
}

